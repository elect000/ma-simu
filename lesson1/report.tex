% Intended LaTeX compiler: pdflatex
\documentclass[a4j, 11pt]{jarticle}
\usepackage[dvipdfmx]{graphicx}
\usepackage[dvipdfmx]{color}
\usepackage{indentfirst}
\usepackage{fancyhdr}
\usepackage{lastpage}
\usepackage{amsmath, amssymb, bm}

\makeatletter
\author{201611350 江畑 拓哉}
\date{2017年10月16日}
\title{演習課題1}

\pagestyle{fancy}

% headers & footers
\lhead{数理アルゴリズム \@title 提出日:\@date\\\@author}
\chead{}
\rhead{}
\lfoot{}
\cfoot{\thepage /\pageref{LastPage}}
\rfoot{}
\renewcommand{\headrulewidth}{0pt}
\renewcommand{\footrulewidth}{0pt}
\makeatother

\begin{document}
\section{課題1}
\label{sec:org1fc4f40}
\subsection{}
\label{sec:orgc44cb96}
以下の命令を実行し,実行結果について説明せよ.\\
\begin{itemize}
\item ones(4, 3)\\
\begin{verbatim}
--> ones(4, 3)
 ans  =

   1.   1.   1.
   1.   1.   1.
   1.   1.   1.
   1.   1.   1.
\end{verbatim}
\end{itemize}
 すべての要素が1の4行3列の行列を作成する。\\
\begin{itemize}
\item eye(5, 3)\\
\begin{verbatim}
--> eye(5, 3)
 ans  =

   1.   0.   0.
   0.   1.   0.
   0.   0.   1.
   0.   0.   0.
   0.   0.   0.
\end{verbatim}
\end{itemize}
 対角要素が1の5行3列の行列を作成する。\\
\begin{itemize}
\item zeros(3, 4)\\
\begin{verbatim}
--> zeros(3,4)
 ans  =

   0.   0.   0.   0.
   0.   0.   0.   0.
   0.   0.   0.   0.
\end{verbatim}
\end{itemize}
 すべての要素が0の3行4列の行列を作成する。\\
\begin{itemize}
\item linspace(-1, 1, 10)\\
\begin{verbatim}
--> linspace(-1,1,10)
 ans  =


         column 1 to 3

  -1.  -0.7777778  -0.5555556

         column 4 to 5

  -0.3333333  -0.1111111

         column 6 to 7

   0.1111111   0.3333333

         column 8 to 10

   0.5555556   0.7777778   1.
\end{verbatim}
\end{itemize}
 -1 から 1 へ向かう等間隔に並んだ10個の点を含んだ一次元配列を作成する。\\
\subsection{}
\label{sec:org54c4f9f}
 v = [1; -2]; w = [1 2]; A = [1 2; 3 4]; を実行し,以下の問いに答えよ.\\
\subsubsection{1-2-1}
\label{sec:orgeb90f04}
 size 関数について以下の命令を実行し,実行結果を比較せよ.\\
\begin{itemize}
\item size(v)\\
\begin{verbatim}
--> size(v)
 ans  =

   2.   1.
\end{verbatim}
\end{itemize}
 2行1列であることを示している。\\
\begin{itemize}
\item size(w)\\
\begin{verbatim}
--> size(w)
 ans  =

   1.   2.
\end{verbatim}
\end{itemize}
 1行2列であることを示している。\\
\begin{itemize}
\item size(A)\\
\begin{verbatim}
--> size(A)
 ans  =

   2.   2.
\end{verbatim}
\end{itemize}
 2行2列であることを示している。\\
\subsubsection{1-2-2}
\label{sec:orgf6d6203}
 max 関数について以下の命令を実行し、実行結果を比較せよ.\\
\begin{itemize}
\item max(A)\\
\begin{verbatim}
--> max(A)
 ans  =

   4.
\end{verbatim}
\end{itemize}
  A のすべての要素に対して最大の要素を返している。\\
\begin{itemize}
\item max(A, 'c')\\
\begin{verbatim}
--> max(A, 'c')
 ans  =

   2.
   4.
\end{verbatim}
\end{itemize}
 Aの各行に対して最大の要素を返している。\\
\begin{itemize}
\item max(A, 'r')\\
\begin{verbatim}
--> max(A, 'r')
 ans  =

   3.   4.
\end{verbatim}
\end{itemize}
 Aの各列に対して最大の要素を返している。\\
\subsubsection{1-2-3}
\label{sec:orgeace88f}
 sum 関数について以下の命令を実行し,実行結果を比較せよ.\\
\begin{itemize}
\item sum(A)\\
\begin{verbatim}
--> sum(A)
 ans  =

   10.
\end{verbatim}
\end{itemize}
 Aのすべての要素を加算した値を返している。\\
\begin{itemize}
\item sum(A, 1)\\
\begin{verbatim}
--> sum(A, 1)
 ans  =

   4.   6.
\end{verbatim}
\end{itemize}
 Aの各行に対して加算した値を返している。\\
\begin{itemize}
\item sum(A, 2)\\
\begin{verbatim}
--> sum(A, 2)
 ans  =

   3.
   7.
\end{verbatim}
\end{itemize}
 Aの各列に対して加算した値を返している。\\
\subsubsection{1-2-4}
\label{sec:orgbead620}
 以下の命令を実行し,実行結果について説明せよ.\\
\begin{itemize}
\item norm(v)\\
\begin{verbatim}
--> norm(v)
 ans  =

   2.236068
\end{verbatim}
\end{itemize}
 そのベクトルのノルムを返している。\\
\begin{itemize}
\item gsort(v)\\
\begin{verbatim}
--> gsort(v)
 ans  =

   1.
  -2.
\end{verbatim}
\end{itemize}
 要素に対してソートを行って、大きい値順に返している。\\
\begin{itemize}
\item abs(v)\\
\begin{verbatim}
--> abs(v)
 ans  =

   1.
   2.
\end{verbatim}
\end{itemize}
 絶対値を取った値を返している。\\
\begin{itemize}
\item inv(A)\\
\begin{verbatim}
--> inv(A)
 ans  =

  -2.    1. 
   1.5  -0.5
\end{verbatim}
\end{itemize}
 逆行列を返している。\\
\section{課題2}
\label{sec:orgdd0afb9}
\subsection{}
\label{sec:org45cc54b}
  \(\bm{A}\) と \(\bm{v}\) を変数 A と v へ代入せよ.\\
\begin{verbatim}
--> A = [4, -2, 0; -1, 4, -2; 0, -1, 4]
 A  = 

   4.  -2.   0.
  -1.   4.  -2.
   0.  -1.   4.
-> v = [3; 0; 1.5]
 v  = 

   3.
   0.
   1.5
\end{verbatim}
\subsection{}
\label{sec:orge422a4d}
  \(\bm{A}\bm{v}\) の計算結果を示せ.\\
\begin{verbatim}
--> A * v
 ans  =

   12.
  -6.
   6.
\end{verbatim}

\subsection{}
\label{sec:org38e22c1}
 ベクトル \(\bm{v}\) の 2 ノルム \(||v||_2\) を求めよ. Scilab の norm 関数を用いてもよい.\\
\begin{verbatim}
-> norm(v)
 ans  =

   3.354102
\end{verbatim}
\subsection{}
\label{sec:org2564028}
 線形方程式 \(\bm{A}\bm{x} = \bm{v}\) の解 \(\bm{x}\) を求めよ.\\
\begin{verbatim}
--> A \ v
 ans  =

   1.
   0.5
   0.5
\end{verbatim}

\section{課題3}
\label{sec:org1070888}
公式の左辺と右辺を計算せよ.\\
\begin{itemize}
\item 値を代入する。\\
\begin{verbatim}
--> A = [1, -1, 0; -1, 2, -1; 0, -1, 2], x = [1;2;3], y = [-2; 2; 1]
 A  = 

   1.  -1.   0.
  -1.   2.  -1.
   0.  -1.   2.

 x  = 

   1.
   2.
   3.

 y  = 

  -2.
   2.
   1.
\end{verbatim}
\item 左辺を計算する。\\
\begin{verbatim}
--> inv(A + x * y')
 ans  =

   5.    0.   -1. 
   3.8   0.2  -0.8
   2.2  -0.2  -0.2
\end{verbatim}
\item 右辺の計算する。\\
\begin{verbatim}
--> inv(A) - (1 / (1 + y' * inv(A) * x))*(inv(A)*x)*(y'*inv(A))
 ans  =

   5.    0.   -1. 
   3.8   0.2  -0.8
   2.2  -0.2  -0.2
\end{verbatim}
\end{itemize}
\section{課題4}
\label{sec:org1448440}
\subsection{}
\label{sec:orgd6e0ba6}
 gsort 関数と abs 関数を用いて, 1 次元配列の絶対値最小の要素と絶対値が 2 番目に小さい要素を返す関数を作成せよ.ただし,返す値は絶対値ではなく,もとの\\
要素の値とすること。\\
\begin{verbatim}
function [val1, val2] = myfunc(vec)
[vals, idxs] = gsort(abs(vec))
idxs = flipdim(idxs, 2)
val1 = vec(idxs(1))
val2 =  vec(idxs(2))
endfunction
\end{verbatim}

\subsection{}
\label{sec:org378b8ef}
\begin{verbatim}
--> [val1, val2] = myfunc(datas)
 val2  = 

  -0.048493

 val1  = 

  -0.0278533
\end{verbatim}
\end{document}
